% Options for packages loaded elsewhere
\PassOptionsToPackage{unicode}{hyperref}
\PassOptionsToPackage{hyphens}{url}
%
\documentclass[
]{book}
\usepackage{lmodern}
\usepackage{amssymb,amsmath}
\usepackage{ifxetex,ifluatex}
\ifnum 0\ifxetex 1\fi\ifluatex 1\fi=0 % if pdftex
  \usepackage[T1]{fontenc}
  \usepackage[utf8]{inputenc}
  \usepackage{textcomp} % provide euro and other symbols
\else % if luatex or xetex
  \usepackage{unicode-math}
  \defaultfontfeatures{Scale=MatchLowercase}
  \defaultfontfeatures[\rmfamily]{Ligatures=TeX,Scale=1}
\fi
% Use upquote if available, for straight quotes in verbatim environments
\IfFileExists{upquote.sty}{\usepackage{upquote}}{}
\IfFileExists{microtype.sty}{% use microtype if available
  \usepackage[]{microtype}
  \UseMicrotypeSet[protrusion]{basicmath} % disable protrusion for tt fonts
}{}
\makeatletter
\@ifundefined{KOMAClassName}{% if non-KOMA class
  \IfFileExists{parskip.sty}{%
    \usepackage{parskip}
  }{% else
    \setlength{\parindent}{0pt}
    \setlength{\parskip}{6pt plus 2pt minus 1pt}}
}{% if KOMA class
  \KOMAoptions{parskip=half}}
\makeatother
\usepackage{xcolor}
\IfFileExists{xurl.sty}{\usepackage{xurl}}{} % add URL line breaks if available
\IfFileExists{bookmark.sty}{\usepackage{bookmark}}{\usepackage{hyperref}}
\hypersetup{
  pdftitle={La feminización del conflicto laboral},
  pdfauthor={Valentina Andrade},
  hidelinks,
  pdfcreator={LaTeX via pandoc}}
\urlstyle{same} % disable monospaced font for URLs
\usepackage{color}
\usepackage{fancyvrb}
\newcommand{\VerbBar}{|}
\newcommand{\VERB}{\Verb[commandchars=\\\{\}]}
\DefineVerbatimEnvironment{Highlighting}{Verbatim}{commandchars=\\\{\}}
% Add ',fontsize=\small' for more characters per line
\usepackage{framed}
\definecolor{shadecolor}{RGB}{248,248,248}
\newenvironment{Shaded}{\begin{snugshade}}{\end{snugshade}}
\newcommand{\AlertTok}[1]{\textcolor[rgb]{0.94,0.16,0.16}{#1}}
\newcommand{\AnnotationTok}[1]{\textcolor[rgb]{0.56,0.35,0.01}{\textbf{\textit{#1}}}}
\newcommand{\AttributeTok}[1]{\textcolor[rgb]{0.77,0.63,0.00}{#1}}
\newcommand{\BaseNTok}[1]{\textcolor[rgb]{0.00,0.00,0.81}{#1}}
\newcommand{\BuiltInTok}[1]{#1}
\newcommand{\CharTok}[1]{\textcolor[rgb]{0.31,0.60,0.02}{#1}}
\newcommand{\CommentTok}[1]{\textcolor[rgb]{0.56,0.35,0.01}{\textit{#1}}}
\newcommand{\CommentVarTok}[1]{\textcolor[rgb]{0.56,0.35,0.01}{\textbf{\textit{#1}}}}
\newcommand{\ConstantTok}[1]{\textcolor[rgb]{0.00,0.00,0.00}{#1}}
\newcommand{\ControlFlowTok}[1]{\textcolor[rgb]{0.13,0.29,0.53}{\textbf{#1}}}
\newcommand{\DataTypeTok}[1]{\textcolor[rgb]{0.13,0.29,0.53}{#1}}
\newcommand{\DecValTok}[1]{\textcolor[rgb]{0.00,0.00,0.81}{#1}}
\newcommand{\DocumentationTok}[1]{\textcolor[rgb]{0.56,0.35,0.01}{\textbf{\textit{#1}}}}
\newcommand{\ErrorTok}[1]{\textcolor[rgb]{0.64,0.00,0.00}{\textbf{#1}}}
\newcommand{\ExtensionTok}[1]{#1}
\newcommand{\FloatTok}[1]{\textcolor[rgb]{0.00,0.00,0.81}{#1}}
\newcommand{\FunctionTok}[1]{\textcolor[rgb]{0.00,0.00,0.00}{#1}}
\newcommand{\ImportTok}[1]{#1}
\newcommand{\InformationTok}[1]{\textcolor[rgb]{0.56,0.35,0.01}{\textbf{\textit{#1}}}}
\newcommand{\KeywordTok}[1]{\textcolor[rgb]{0.13,0.29,0.53}{\textbf{#1}}}
\newcommand{\NormalTok}[1]{#1}
\newcommand{\OperatorTok}[1]{\textcolor[rgb]{0.81,0.36,0.00}{\textbf{#1}}}
\newcommand{\OtherTok}[1]{\textcolor[rgb]{0.56,0.35,0.01}{#1}}
\newcommand{\PreprocessorTok}[1]{\textcolor[rgb]{0.56,0.35,0.01}{\textit{#1}}}
\newcommand{\RegionMarkerTok}[1]{#1}
\newcommand{\SpecialCharTok}[1]{\textcolor[rgb]{0.00,0.00,0.00}{#1}}
\newcommand{\SpecialStringTok}[1]{\textcolor[rgb]{0.31,0.60,0.02}{#1}}
\newcommand{\StringTok}[1]{\textcolor[rgb]{0.31,0.60,0.02}{#1}}
\newcommand{\VariableTok}[1]{\textcolor[rgb]{0.00,0.00,0.00}{#1}}
\newcommand{\VerbatimStringTok}[1]{\textcolor[rgb]{0.31,0.60,0.02}{#1}}
\newcommand{\WarningTok}[1]{\textcolor[rgb]{0.56,0.35,0.01}{\textbf{\textit{#1}}}}
\usepackage{longtable,booktabs}
% Correct order of tables after \paragraph or \subparagraph
\usepackage{etoolbox}
\makeatletter
\patchcmd\longtable{\par}{\if@noskipsec\mbox{}\fi\par}{}{}
\makeatother
% Allow footnotes in longtable head/foot
\IfFileExists{footnotehyper.sty}{\usepackage{footnotehyper}}{\usepackage{footnote}}
\makesavenoteenv{longtable}
\usepackage{graphicx,grffile}
\makeatletter
\def\maxwidth{\ifdim\Gin@nat@width>\linewidth\linewidth\else\Gin@nat@width\fi}
\def\maxheight{\ifdim\Gin@nat@height>\textheight\textheight\else\Gin@nat@height\fi}
\makeatother
% Scale images if necessary, so that they will not overflow the page
% margins by default, and it is still possible to overwrite the defaults
% using explicit options in \includegraphics[width, height, ...]{}
\setkeys{Gin}{width=\maxwidth,height=\maxheight,keepaspectratio}
% Set default figure placement to htbp
\makeatletter
\def\fps@figure{htbp}
\makeatother
\setlength{\emergencystretch}{3em} % prevent overfull lines
\providecommand{\tightlist}{%
  \setlength{\itemsep}{0pt}\setlength{\parskip}{0pt}}
\setcounter{secnumdepth}{5}
\usepackage{booktabs}
\usepackage[]{natbib}
\bibliographystyle{apalike}

\title{La feminización del conflicto laboral}
\usepackage{etoolbox}
\makeatletter
\providecommand{\subtitle}[1]{% add subtitle to \maketitle
  \apptocmd{\@title}{\par {\large #1 \par}}{}{}
}
\makeatother
\subtitle{Un análisis temporal y comparado de la densidad sindical mundial}
\author{Valentina Andrade}
\date{22 agosto 2020}

\begin{document}
\maketitle

{
\setcounter{tocdepth}{1}
\tableofcontents
}
\hypertarget{presentaciuxf3n}{%
\chapter*{Presentación}\label{presentaciuxf3n}}
\addcontentsline{toc}{chapter}{Presentación}

Presentación y agradecimientos

\hypertarget{resumen}{%
\chapter*{Resumen}\label{resumen}}
\addcontentsline{toc}{chapter}{Resumen}

La feminización de la sindicalización hoy es un fenómeno mundial. Distintas investigaciones han abordado el problema de la sindicalización femenina desde factores culturales que redundan en la pasividad de las mujeres y exclusión de estas en la acción colectiva. Ante la insuficiencia de poder explicar por qué hoy hay más mujeres en los sindicatos, la siguiente investigación pone énfasis en las determinantes cíclicas, estructurales e institucionales de los cambios de la sindicalización. El aumento en la participación laboral femenina, la feminización de ciertas industrias, los cambios en las brechas salariales, aumento del empleo parcial y los distintos arreglos institucionales son algunos de las determinantes a considerar en un análisis de series temporales comparativo entre 56 países. Con la construcción de una \href{../output/data}{base de datos} propia con cobertura de 1980-2018 cuyas principales fuentes de información son encuestas de hogares, datos administrativos, ICTWSS, ILO, Banco Mundial y OCDE, se estimaron modelos de corrección de error y usando efectos fijos se distinguió entre efectos a largo plazo y a corto plazo de los determinantes.

\textbf{Palabras claves}: sindicalización, feminización, mercado laboral, salarios, tiempos de trabajo

\hypertarget{intro}{%
\chapter{¿Más mujeres en los sindicatos?}\label{intro}}

\hypertarget{el-problema-sobre-la-sindicalizaciuxf3n-de-mujeres}{%
\section{El problema sobre la sindicalización de mujeres}\label{el-problema-sobre-la-sindicalizaciuxf3n-de-mujeres}}

Las trabajadoras del aseo, las teleoperadoras de los centros de llamado, las trabajadoras de los hospitales, las maestras, las bancarias, las vendedoras del comercio, las cajeras, las cocineras, las trabajadoras domésticas. Hoy, todas ellas son fuerza de trabajo de los servicios. Antes, probablemente hijas y nietas de lavanderas, obreras textiles o dueñas de casa. El evidente desplazamiento hacia la tercerización y feminización de la fuerza de trabajo deja abierta la pregunta a si estos cambios tuvieron un correlato en el conflicto laboral y en la formación de la clase trabajadora. ¿Es posible pensar que hoy hay más mujeres en los sindicatos que en décadas anteriores?

Antes de sumar uno más uno -- o, en otras palabras, determinar como causa de la mayor participación sindical de mujeres su mayor participación en el mercado de trabajo-- no es menor destacar que lo que llama la atención de la crisis de la sociedad industrial es la supuesta muerte anunciada del proletariado. Las relaciones laborales habrían cambiado a tal punto que el antagonismo entre la clase capitalista y los trabajadores se habría transformado radicalmente: el fin del obrero hombre asalariado clásico sería en fin de la organización de la clase trabajadora tal como la conocíamos (Pakulski \& Waters, 1996). Porque la desindustrialización sería el fin del trabajo, y con ello, el fin de la clase trabajadora y su acción colectiva. En consecuencia, los servicios no serían un trabajo, sus empleadas no serían parte de una clase, y mucho menos tendrían capacidad de organización.

Para los expertos en relaciones laborales, en ningún el desplazamiento de los hombres desde las principales fuentes de trabajo implicó que, más mujeres se insertarían en el mercado laboral, por lo que el razonamiento de que ellas ahora serían las principales actoras sindicales ni si quiera aparece. Más bien, las investigaciones se han centrado en el declive de la acción sindical (Vachon et al., 2016; Western, 1995a). Pencavel (2005), en esta línea, muestra como a nivel global se dan una serie de transformaciones que parecen debilitar el poder de los sindicatos. Dentro de estas se ponen de relieve las variaciones en la composición del empleo, como el aumento de trabajadores \emph{white collars} y el aumento de la participación femenina en el empleo de la mano con el trabajo \emph{part time}. También, se señala la reestructuración hacia mercados de trabajo más internacionalizados, liberalizados, con crecimiento económico menos acelerado, que han producido mayores tasas de desempleo e informalidad (Blanchard \& Wolfers, 2000; Vachon et al., 2016). Y, por último, se ha señalado la implementación de políticas laborales cada vez menos amigables con el sindicalismo (Disney, 1990), como la descentralización de las negociaciones colectivas (Calmfors \& Driffill, 1988), la eliminación de los \emph{closed shops} (Ebbinghaus et al., 2011; Pencavel, 2004) y la desregulación del empleo (Vallas \& Beck, 1996).

A partir de la década del noventa, el diagnóstico del descenso de la afiliación sindical a nivel mundial se volvió una realidad indiscutida en la literatura especializada: en 1978 casi la mitad de los asalariados participaba en un sindicato, mientras que treinta años después la sindicalización solo alcanzó a un 23\% (Bryson et al., 2011). Es amplia la evidencia que señala que esta baja en la sindicalización fue un fenómeno que no discriminó entre países desarrollados y en vías de desarrollo\footnote{Blanchfower (2006, p.~14) expone que países como Estados Unidos, Inglaterra, Australia, Alemania, Japón, Países Bajos, Nueva Zelanda, Argentina, Brasil, Bolivia, Perú, México, Taiwán, Hungría, Polonia. República Checa y Eslovenia son países donde disminuye la sindicalización. Mientras que hay aumento en Finlandia, Islandia, Dinamarca, Bélgica, Suecia, Chile, India, Turquía, Sudáfrica, Costa Rica, El Salvador y República Dominicana. Se mantienen estables Canadá, Noruega, España, Italia, Panamá y Filipinas.} (Blanchflower, 2006).

El centro fue ese desoldador escenario, donde los estudios laborales se volcaron explicar la erosión del sindicalismo a nivel mundial\footnote{Por citar algunos ejemplos están los trabajos de Charlwood \& Haynes (2008) sobre Nueva Zelanda, Clawson \& Clawson (1999) y Farber (1990) sobre Estados Unidos, Visser (2007) analizando el caso alemán, Machin (2000) sobre el Reino Unido y más recientemente Palm (2017) en su análisis sobre Suecia. Para una perspectiva comparada, Western (1995b) muestra el declive del sindicalismo en 18 países.}. Se esbozaron una serie de críticas sobre la capacidad de los sindicatos de representar al nuevo trabajador marginal, no calificado y sin contrato (Visser, 2012). Pero este enfoque no solo se dio desde las llamadas investigaciones ``ciegas al género''. Incluso la negación a la acción colectiva de mujeres ha llegado a tal punto que las estudiosas sobre sindicatos y mujeres se han centrado en explicar la exclusión y marginación de estas (Díaz, 2005; Godinho-Delgado, 1990; Healy \& Kirton, 2013; Kirton, 2005) más que su posible mayor participación en el conflicto laboral o las particulares formas en las que han hecho.

Pero así como el fin de la clase trabajadora fue un precipitado augurio, el declive del sindicalismo puede ser puesto en tela de juicio, pues a partir de los 2000 nuevas tendencias y oportunidades han comenzado a emerger (Artus \& Pflüger, 2015): contrario a lo que se había señalado hasta ahora en la literatura tradicional sobre sindicalización\footnote{Son escasos los estudios que pudieron dar luces de este nuevo fenómeno. Machin (2004, p.~430) señala que en Inglaterra ``para el 2001 ya no había brecha de género dentro de la membresía sindical'', mientras que Blanchflower (2006) constata que las mujeres tienen una significativa densidad sindical en países del ex bloque soviético (Latvia, Polonia, Rusia, Eslovaquia y Eslovenia), países escandinavos, Suecia, Nueva Zelanda, Chile, Israel, entre otros. Sin embargo, ambos lo muestran como excepciones a nivel internacional. Recientemente Hassel \& Schroeder (2019) y Visser (2019) confirman esta posibilidad.}, en diversos países no solo hay más mujeres en los sindicatos, sino que incluso se ha feminizado la densidad sindical. El siglo XXI se inaugura con un nuevo sindicalismo, el sindicalismo de mujeres.

Pero así como el fin de la clase trabajadora fue un precipitado augurio, el declive del sindicalismo puede ser puesto en tela de juicio, pues a partir de los 2000s nuevas tendencias y oportunidades han comenzado a emerger (Artus \& Pflüger, 2015): contrario a lo que se había señalado hasta ahora en la literatura tradicional sobre sindicalización , en diversos países no solo hay más mujeres en los sindicatos, sino que incluso se ha feminizado la densidad sindical. El siglo XXI se inaugura con un nuevo sindicalismo, el sindicalismo de mujeres.

Ciertamente en la mayoría de los países los hombres tienen en términos absolutos una participación mayor en las organizaciones de trabajadores (Visser, 2015), en tanto su participación en el mercado de trabajo es mayor. Pero si ponemos a las mujeres sindicalizadas en proporción a cuántas de estas participan en el mercado laboral se puede notar con creces que a nivel mundial las mujeres progresivamente han aumentado su participación sindical (Hassel \& Schroeder, 2019; Visser, 2019). En otras palabras, la densidad sindical de las mujeres ha aumentado (Figura 1.1). La feminización de la sindicalización, es decir, donde la proporción de sindicalización femenina ha superado a la masculina, se puede observar en países tan diversos como Inglaterra, Suecia, Noruega, Estonia, Finlandia, Canadá, Australia, Rusia (y los países excomunistas en general), Islandia, Nueva Zelanda, Dinamarca, México, Brasil y Chile, los cuáles se presentan a modo de ejemplos el Mapa 1. O incluso, en países como Estados Unidos la brecha de participación sindical por sexo ha disminuido año a año (Milkman, 2007). De hecho, resulta más sencillo y simple notar las excepciones de países donde la masculinización de la sindicalización sigue vigente: Francia, Alemania, Corea, Japón, Bélgica, Austria, Colombia y Argentina (Figura 1.2).

\begin{Shaded}
\begin{Highlighting}[]
\CommentTok{# FUDi (Feminization Union Density Index) se calculó como la relación entre densidad sindical femenina y densidad sindical masculina por país, en el último año disponible. Los valores cercanos a 1 indican igualdad en la distribución sexual de la sindicalización, mientras que valores mayores indican feminización de la sindicalización}
\end{Highlighting}
\end{Shaded}

En cierto modo, la tendencia de la feminización sindical es visible una vez que se pone en relación y perspectiva con la inserción de las mujeres en el mercado laboral. Este ajuste dista de ser antojadizo. Más bien, diversos estudios sobre la membresía sindical lo han puesto como un común denominador para discutir los determinantes de la sindicalización (Schnabel, 2002, 2013) La razón, es que cambios grandes en la densidad sindical generalmente indican un cambio en las relaciones laborales y una evolución radical de los mercados de trabajo (Bryson et al., 2011, p.~98).

Pero como se señaló, las investigaciones sobre la participación sindical femenina han puesto su foco más en la exclusión. Por un lado, las estructuralistas argumentan que el patriarcado es el aspecto que determina la subordinación de las mujeres dentro de las organizaciones de trabajadores (Hartmann, 1976). Por otro, las culturalistas plantean que existe una ``cultura masculina'' en el sindicalismo lo que aleja a las mujeres de su participación (Kirton \& Greene, 2005; Ledwith, 2012). Evidentemente estos enfoques hacen difícil explicar los cambios históricos en el movimiento de los trabajadores (Milkman, 1990), como la evidente feminización de la sindicalización registrada en los último años (Mapa 1). Estos enfoques presuponen que el interés de género de las trabajadoras tiene un mayor peso que el interés de la clase (Milkman, 1987) y reifican los roles de género, donde las mujeres aparecen solo orientadas a la familia y pasivas frente a posibles condiciones de deprivación económica.

En oposición a estas perspectivas, la siguiente investigación tiene como propósito contribuir a la discusión del sindicalismo de mujeres no desde los enfoques de género, sino que desde determinantes menos tradicionales y directamente vinculadas al mundo del trabajo y la producción. Esto, ante la incapacidad de ser teorías explicativas de la feminización de la densidad sindical. Los factores analizados provienen de un cuerpo teórico y empírico que ha tenido resultados sustantivos a la hora de explicar tendencias mundiales y diferencias entre países (Schnabel, 2013), a saber, nos referimos al impacto de determinantes como los (1) ciclos económicos, (2) estructura productiva e (3) instituciones laborales, que tienen sobre el mercado laboral y los sindicatos, pero poniendo especial énfasis a cómo estos podrían haber afectado especialmente a las mujeres trabajadoras y su relación con los sindicatos. Esto bajo la noción planteada por Visser (2012, p.~100) quien señala que los cambios en el mercado laboral no son neutrales, sino que tienen efectos selectivos: no solo influencian el nivel -y probabilidad- de sindicalización, además afectan quienes se unen a organizaciones sindicales y quienes no.

Por un lado, los ciclos económicos se entienden como aquellas fluctuaciones a corto plazo que contienen periodos de auge (eg. crecimiento de la productividad y empleo ) y declive (eg. contracción por altas tasas de desempleo), y que se ha identificado que estas tienen un carácter procíclico con la sindicalización (Bain \& Elsheikh, 1976; Calmfors et al., 2001; Riley, 1997). Nuestra propuesta es que, si se miran las crisis económicas, al contrario de lo que ocurre con la tendencia general, las mujeres aumentarán su empleabilidad, pero asociado a puestos precarios y con brechas salariales altas, lo que las llevará a sindicalizarse como un modo de resguardar sus condiciones laborales. En consecuencia, se prueba la hipótesis de que un aumento en el desempleo femenino feminiza la densidad sindical debido a la inclusión de más mujeres en los sindicatos, pero también al escape de los hombres de estas por el aumento de su desempleo.

Por otro lado, se profundiza en como los cambios estructurales y a largo plazo tales como transformaciones en la estructura productiva y de composición del empleo tienen sobre la feminización de la densidad sindical. Se propone la existencia de una convergencia entre el crecimiento de las pink collars - esto es crecimiento de sectores feminizados como el sector servicios y el empleo part time -- con la feminización de la densidad sindical. La propuesta es que, con el crecimiento de los sectores económicos feminizados, a largo plazo se producirá un cambio desde una estructura sindical masculina y tradicional hacia una estructura sindical dominada por esta nueva categoría laboral. Es decir, la feminización y tercerización de la densidad sindical (Artus \& Pflüger, 2015, p.~93; Kocsis et al., 2013) convergen ante un escape de los hombres de las estructuras sindicales y un aumento de las mujeres en estas, principalmente justificado por la reestructuración que se viven en los mercados laborales a partir de los 80.

Por último, se reconoce que la feminización de la densidad sindical ha ocurrido en algunos países y en otros no. Si bien una parte de ello se puede deber a las diferencias entre los países sobre los factores antes mencionados, es de esperar que la feminización de la densidad sindical esté asociada al marco institucional en el que se desenvuelven las relaciones laborales, bajo el entendido de que las políticas laborales -- políticas salariales, negociación colectiva, políticas de protección- producen contextos favorables o desfavorables para el reclutamiento sindical (Alemán, 2009; Calmfors et al., 2001; Crouch, 1982; Ebbinghaus et al., 2011) . Se propone que, en países de economías centralizadas, donde hay más coordinación salarial y extensión de beneficios de la negociación colectiva, el aumento de la participación laboral femenina no necesariamente significará una feminización de la sindicalización pues esta no será condición necesaria para recibir los derechos colectivos.

En consecuencia, los tres determinantes económicos y laborales analizados se examinan en su ``genderización'': no solo se trata observar los aumentos del empleo, sino que un aumento en la participación laboral femenina o masculina; no es solo la cuestión salarial, sino la desigualdad salarial entre géneros; no es solo la cuestión de los tiempos de trabajo, sino que la distribución de las jornadas para enfrentar también el trabajo doméstico. Estos ``puntos críticos de género'' (Cox et al., 2007, p.~719) son dignos de ser examinados como preocupaciones en donde las trabajadoras pueden exigir una mayor participación sindical. Principalmente, esta es la razón de porqué nos acercamos más determinantes de la economía laboral que a cuestiones de la cultura. Pero, como anunciamos, los cambios en el mercado laboral afectarán de manera distinta a hombres y mujeres: las condiciones de explotación de las mujeres se mueven en una ecuación que contiene tanto el proceso de producción como el de reproducción (Federici, 2018). Es por ello que, se observa la feminización de la sindicalización desde una mirada marxista-feminista pues para entender la organización de la clase trabajadora femenina se está poniendo énfasis a las condiciones materiales en las que las mujeres se insertan en esta ecuación producción-reproducción: se analiza su participación en el empleo parcial, su desempleo, su condición salarial y su poder de negociación bajo ciertos sistemas de relaciones laborales.

La tesis central que se busca defender es que las mujeres trabajadoras necesitan a los sindicatos y los sindicatos necesitan a las mujeres trabajadoras. Hoy, la fuerza laboral del sindicalismo es la mujer sindicalista. En otras palabras, lo que ha traído a las mujeres a la organización sindical tiene que ver más con la ubicación que la mujer han empezado a ocupar en el mercado laboral (i.e su predominancia en el sector servicios y en empleos flexibles) y en como el sindicalismo les sirve como motor para canalizar sus demandas; como también ocurre que los sindicatos han necesitado de esta nueva fuerza de trabajo para mantenerse vivos.

El modo por el cual se ha abordado la tesis de la feminización de la sindicalización implica una perspectiva temporal y comparada, esto es, un análisis de series de tiempo entre países. Sobre el primer término, la densidad sindical por sexo no siempre ha sido igual, y como apunta Milkman (1990), la única forma de capturar la transformación de los intereses de las asalariadas y su consiguiente sindicalización, es analizando largos procesos históricos que develen los cambios en las condiciones de explotación de las mujeres trabajadoras. Igualmente, la feminización sindical es un fenómeno reciente, por lo que también se debe puntualizar cómo determinados procesos económicos han incidido en el corto plazo en este cambio. Sobre el segundo término, la feminización de la sindicalización es un fenómeno que ha ocurrido en ciertos países y en otros no, por lo que ha sido necesario comparar las condiciones institucionales y económicas de las naciones y ver cómo estas inciden en las mujeres (O'Reilly \& Fagan, 1998). En consecuencia, la pregunta y objetivos que esta investigación busca responder es:

\hypertarget{pregunta-de-investigaciuxf3n}{%
\section{Pregunta de investigación}\label{pregunta-de-investigaciuxf3n}}

¿Cuál es la relación entre los cambios en la densidad sindical por sexo y determinantes económicos-laborales a nivel mundial entre 1980-2018?

\hypertarget{objetivo}{%
\section{Objetivo}\label{objetivo}}

Analizar la relación entre los cambios en la densidad sindical por sexo y determinantes económicos-laborales a nivel mundial entre 1980-2018

\hypertarget{objetivos-especuxedficos}{%
\section{Objetivos específicos}\label{objetivos-especuxedficos}}

\emph{O.E.1} -- Relacionar los cambios en la densidad sindical por sexo y la estructura de empleo a nivel mundial entre 1980-2018

\emph{O.E.2} - Relacionar los cambios en la densidad sindical por sexo y los ciclos económicos a nivel mundial entre 1980-2018

\emph{O.E.3} - Relacionar los cambios en la densidad sindical por sexo y las instituciones laborales a nivel mundial entre 1980-2018

\hypertarget{estructura-de-la-investigaciuxf3n}{%
\section{Estructura de la investigación}\label{estructura-de-la-investigaciuxf3n}}

La presente investigación se organiza en los siguientes capítulos temáticos. El \protect\hyperlink{intro}{capítulo 1 ¿Más mujeres en los sindicatos?} que recién presentado, considera el resumen, el desarrollo del problema de investigación, presenta la tesis principal, la pregunta y los objetivos generales y específicos de la investigación.

El \protect\hyperlink{cap2}{capítulo 2 ¿Por qué ellas hoy son más y ellos menos?} construye un marco analítico para la densidad sindical por sexo a partir de una síntesis de los antecedentes empíricos y teóricos del problema de la sindicalización en general y en particular para las mujeres. La exposición de este capítulo se basa en una reconstrucción de los cambios económicos e institucionales del capitalismo desde 1960 a la fecha. Se vincularon esos cambios a las transformaciones en densidad sindical y a qué explicaciones se dieron tanto a los auges como declives sindicales. Se formalizaron tres grandes determinantes: los ciclos económicos, estructurales e institucionales, y a partir de ellos, se revisó cómo estos cambios podrían haber afectado la relación entre las mujeres y sindicatos. En consecuencia, este capítulo cierra con las hipótesis sustantivas sobre la feminización de la densidad sindical femenina

El \protect\hyperlink{cap3}{capítulo 3} presenta los datos y variables utilizados. Asimismo, se formaliza el análisis de series temporales entre países a utilizar, esto es, un modelo de correción de error (ECM) que permite estimar efectos a largo y corto plazo de las determinantes sobre cada uno de los países.

\hypertarget{por-quuxe9-ellas-hoy-son-muxe1s-y-ellos-menos-hacia-un-marco-analuxedtico-de-la-densidad-sindical-por-sexo}{%
\chapter{¿Por qué ellas hoy son más y ellos menos?: hacia un marco analítico de la densidad sindical por sexo}\label{por-quuxe9-ellas-hoy-son-muxe1s-y-ellos-menos-hacia-un-marco-analuxedtico-de-la-densidad-sindical-por-sexo}}

\emph{``Working women needs unions, and unions needs working women: today, union labor force is union working women''}

Desde finales de 1980 la reestructuración productiva, desregulación de los mercados de trabajo, la internacionalización y los altos niveles de desempleo tuvieron profundas implicancias en la participación laboral y sindical de las mujeres. A la luz de estos fenómenos se produjo un evidente crecimiento de las mujeres en el mercado de trabajo, especialmente con una importante participación en el sector servicios y el empleo a tiempo parcial. Ahora bien, gran parte de la literatura ha puesto atención a los factores antes mencionados como erosionadores del sindicalismo, pasando por alto que estos cambios coinciden con la creciente importancia que han ganado a nivel mundial las mujeres en los sindicatos: hoy más que nunca hay más mujeres en los sindicatos, y la diferencia de sindicalización entre hombres y mujeres ha desaparecido en gran parte de los países del mundo.
Este capítulo buscará llenar ese vacío examinando el impacto de estos determinantes- la reestructuración productiva, los ciclos económicos y los cambios institucionales en las relaciones industriales- sobre el mercado laboral y los sindicatos, pero poniendo especial énfasis a porqué esto podría haber afectado a las mujeres trabajadoras y su relación con los sindicatos. La tesis central que se busca defender, ante la evidente convergencia hacia la feminización de la sindicalización, ya ha sido anunciada: las mujeres trabajadoras necesitan a los sindicatos y los sindicatos necesitan a las mujeres trabajadoras. Hoy, la fuerza laboral del sindicalismo es la mujer sindicalista. En otras palabras, lo que ha traído a las mujeres a la organización sindical tiene que ver más con la ubicación que la mujer han empezado a ocupar en el mercado laboral (i.e su predominancia en el sector servicios y en empleos flexibles) y en como el sindicalismo les sirve como motor para canalizar sus demandas; como también ocurre que los sindicatos han necesitado de esta nueva fuerza de trabajo para mantenerse vivos.
En consecuencia, la estructura argumentativa de la tesis retoma estos elementos a partir de un análisis empírico y teórico de: (1) cómo se evidencian los cambios y diferencias de densidad sindical entre los países a partir de 1960 a la actualidad, como una forma de contextualización general de los cambios recientes en el capitalismo y los sindicatos; (2) se plantean factores como los ciclos económicos, en parte pues estos han permitido explicar los crecimientos sindicales; (3) se abordan determinantes como los cambios en la estructura productiva y composición de la fuerza de trabajo debido a que no solo son factores que abordaron el declive sindical sino que también son la base para entender como se altera el ambiente para la organización de las mujeres en el trabajo; (4) se puntualizan cómo los cambios en la densidades sindicales presentan divergencias en la última década, explicado a partir de las diferencias institucionales en los países. Abordar estas relaciones, sobre todo a razón de cómo ha afectado a la densidad sindical femenina, es esencial en tanto estas teorías si bien han sido limitadas en explicar la sindicalización de las mujeres, si han tenido respuestas satisfactorias al entendimiento de los patrones de densidad sindical al menos de los países centrales. Como consecuencia de ello, el capítulo cierra indicando esta ausencia en el análisis de la densidad sindical por sexo, mostrando algunos estudios empíricos que han sido la excepción a la regla. A partir de esta discusión, se propone un marco analítico para la densidad sindical por sexo proponiendo hipótesis de porqué la feminización de la sindicalización ha emergido en algunos países y en otros no.

\hypertarget{una-perspectiva-histuxf3rica-y-comparada-de-la-densidad-sindical-en-el-mundo}{%
\section{Una perspectiva histórica y comparada de la densidad sindical en el mundo}\label{una-perspectiva-histuxf3rica-y-comparada-de-la-densidad-sindical-en-el-mundo}}

Entre los distintos países del mundo los porcentajes de trabajadores que forman parte de una organización sindical presentan diferencias abismales (ver Figura 2.1). Así mismo, esta densidad sindical, en un mismo país, contrasta notablemente entre los periodos de su historia.

\begin{Shaded}
\begin{Highlighting}[]
\CommentTok{# Figura Figura 2.1 Densidad sindical en países OCDE para último año (2016)}
\CommentTok{#Fuente: Elaboración propia en base a ICTWSS (2019) con el último año de densidad sindical registrado para cada país. }
\end{Highlighting}
\end{Shaded}

La literatura sobre la densidad sindical ha observado tres grandes movimientos históricos del fenómeno. El primero es un periodo de auge de la sindicalización, que tiene sus orígenes en los finales de la Gran Guerra y, su apogeo y crisis, a inicios de los años setenta, con la crisis del petróleo, el fin de la convertibilidad del oro y las movilizaciones obreras y estudiantiles del 68' (Silver, 2003; Traxler et al., 2001; Tronti, 1966 {[}2001{]}). Las bases de este crecimiento sindical se encuentran en los grandes compromisos de clase a nivel institucional que sostuvieron los sectores organizados de las clases medias y obreras, con representantes del capital y gobiernos de corte socialdemócrata (Korpi, 1983). Esto, en el marco de un desempeño económico virtuoso que permitía conciliar crecientes demandas salariales y transferencias en servicios sociales con óptimas tasas de acumulación e inversión . Es durante este periodo que historiadores y economistas buscaron explicar este crecimiento sindical, principalmente poniendo el foco en macro determinantes como los ciclos económicos, tales como la inflación y el desempleo.
El segundo es un periodo de drástica caída de la sindicalización (Western, 1995) y de la actividad huelguística (Shalev, 1992), cuyas raíces han sido identificadas en los procesos de globalización (Western y Wallerstein, 2000) y reestructuración de los procesos productivos (Jenkins y Leich, 1997), que producirán cambios y presiones sin precedentes sobre la organización del trabajo durante los años ochenta y noventa. Los modelos que emergen este periodo se centrarán en estudiar las determinantes a nivel estructural de la densidad sindical, ya sea en factores como la estructura productiva o en la composición del empleo.
Por último, desde los años dos mil se identifican grandes divergencias entre y dentro de los países (Visser, 2019), lo que lleva a observar dispares procesos de organización y movilización entre sectores de la clase trabajadora. Así, conviven teorías que insisten en evidenciar una convergencia hacia un declive sindical producto de la neoliberalización económica (Baccaro y Howell, 2011), mientras que otras identifican resultados distintos como procesos de revitalización sindical (Kelly \& Frege, 2004) y asensos de la conflictividad laboral en industrias y regiones particulares del sistema mundo (Silver, 2003). Siguiendo y ampliando este argumento, esta investigación sostiene que es discutible si existe o no una convergencia hacia la feminización de la sindicalización: mientras la densidad sindical de los hombres se encuentra estancada o en franca retirada, la feminización sindical toma protagonismo e invita a una relectura de aquello que parecía finalizado.

\begin{Shaded}
\begin{Highlighting}[]
\CommentTok{#Figura Figura 2.2 Densidades sindicales por continente (1960-2020)}

\CommentTok{#ICTWSS}
\end{Highlighting}
\end{Shaded}

En la Figura 2.2 se pueden observar los tres periodos mencionados de aumento, crisis y divergencia en los países centrales, como los son los de Europa Occidental y Estados Unidos. Entre 1960 a 1980, a excepción de los Países Bajos y Francia, gran parte de Europa presenta un aumento en su densidad sindical. La misma tendencia se puede identificar en países de otros continentes como Canadá y Nueva Zelanda. A finales de 1980, con la introducción del neoliberalismo e internacionalización de las relaciones laborales, la tasa de sindicalización declinó sin discriminación en los países: en los cuatro países nórdicos (Suecia, Noruega, Finlandia y Dinamarca) esta disminuyó de un 90\% a 68\%; en Europa continental del Oeste (Austria, Bélgica, Francia, Alemania, Italia, Países Bajos, Portugal, España y Suecia) de 43\% a 24\% (Visser, 2019). Un ejemplo muy claro es Nueva Zelanda: en inicio de la década de 1960 el país bordeaba una sindicalización de un 45,8\%, a 1980 llega a 69\%, y en menos de 10 años el país pierde en densidad sindical todo lo que había ganado en los años precedentes. Evidentemente este análisis no es igual al introducir naciones pertenecientes a continentes como Sudáfrica, Asia, América Latina y Europa del Este. La periodización clásica se torna confusa y los tres grandes movimientos de la densidad sindical dejan de ser claros. Por un lado, podremos notar que en Asia las trayectorias son disímiles en el sentido que, si bien Rusia e Israel siguen la tendencia de Europa Central, países como China viven un auge en la sindicalización durante los 80 y luego de la crisis del 2008. Por otro, países latinoamericanos como Chile y Argentina muestran un auge en la sindicalización al menos hasta los 80 -- hasta el 73 en el caso chileno -, pero luego se suman a la tendencia de declive generalizado que se vive en países como México, Brasil y Estados Unidos .\\
Pero ¿qué ocurre específicamente con la sindicalización femenina? Conocidos son los estudios que sugieren que las mujeres tienen aversión a los sindicatos debido a que, por un lado, estos son incapaces de promover sus ``intereses de género'' (Greene \& Kirton, 2006; Sinclair, 1995; Walters, 2002), y por otro, las mujeres tendrían actitudes menos militantes y de confrontación que hombres, por lo que el sindicalismo no sería un espacio para ellas (Tomlinson, 2005). Las causas estén o en la organización sindical o en las actitudes de las mujeres, lo cierto es que las investigaciones han consensuado que la menor participación de las mujeres en los sindicatos está atravesada también por el declive del sindicalismo.
Sorprendentemente es una tendencia que se ha revertido. A partir de los años 2000 nuevas oportunidades han comenzado a emerger en la organización de los trabajadores: en diversos países no solo hay más mujeres en los sindicatos, sino que también la sindicalización femenina ha ido superando a la masculina - lo que llamaremos feminización del sindicalismo. Efectivamente en la mayoría de los países los hombres tienen en términos absolutos una mayor participación en los sindicatos (Schnabel, 2013a; Visser, 2015) pero esto se produce en la medida en que su participación en la fuerza de trabajo es mayor.
Sin embargo, como anunciábamos al inicio de esta investigación, si controlamos por la participación en el mercado laboral, se puede notar con creces que a nivel mundial las mujeres han aumentado su densidad sindical. De los 25 países en las gráficas, en 14 de ellos se ha feminizado el conflicto laboral (Figura 2.3): Australia, Canadá, Chile, Dinamarca, Estonia, Finlandia, Islandia, México, Nueva Zelanda, Noruega, Rusia, Suecia y el Reino Unido. Mientras que en los países restantes se puede evidenciar que si bien la densidad sindical femenina no ha superado a la masculina han disminuido notablemente las brechas de sindicalización por género -como en Estados Unidos- y/o la sindicalización masculina ha ido disminuyendo. El rápido avance de la feminización del sindicalismo, combinado con la caída de la sindicalización masculina, es probablemente la ``mayor y más profunda transformación en el sindicalismo'' (Visser, 2006, p.~47)

\begin{Shaded}
\begin{Highlighting}[]
\CommentTok{#Figura 2.3 Densidad sindical feminizada por género y países, 1960-2018}
\CommentTok{#Fuente: Elaboración propia en base a Encuestas de Hogares por país}
\end{Highlighting}
\end{Shaded}

\begin{Shaded}
\begin{Highlighting}[]
\CommentTok{#Figura 2.4 Densidad sindical por género y país, 1960-2018}
\CommentTok{#Fuente: Elaboración propia en base a Encuestas de Hogares por país.}
\end{Highlighting}
\end{Shaded}

Incluso, en alguno de los países donde se ha feminizado la densidad sindical -la participación de las mujeres en el sindicalismo, en relación con su participación en la fuerza de trabajo- también se ha feminizado la proporción sindical -la participación de las mujeres en los sindicatos, en relación con el total de sindicalizados.

\begin{Shaded}
\begin{Highlighting}[]
\CommentTok{#Figura 2.5 Proporción de mujeres y hombres en los sindicatos, 2011}
\CommentTok{#Fuente: Elaboración propia en base a replicación de ILO (2011)}
\end{Highlighting}
\end{Shaded}

La feminización del conflicto laboral es una profunda transformación que, como muestran las cifras, parece no referir a una variación particular de un país: si Suecia de 1983 inauguró el fenómeno, la mayoría de los países evidencian este cambio en los años 2000, coincidentemente cuando se comenzaron a sentir los efectos de la desindustrialización. Así también, dista de ser una alteración con pronto retorno pues las cifras muestran que la tendencia o se ha estabilizado -como el Reino Unido y Suecia- o incluso aumentado la brecha de sindicalización por género, a favor de las mujeres -como en Chile. Ahora bien, \emph{¿cómo se pueden explicar estos patrones de densidad sindical por sexo?}

\hypertarget{determinantes-cuxedclicos-del-crecimiento-sindical-un-anuxe1lisis-del-empleo-y-los-salarios}{%
\section{Determinantes cíclicos del crecimiento sindical: un análisis del empleo y los salarios}\label{determinantes-cuxedclicos-del-crecimiento-sindical-un-anuxe1lisis-del-empleo-y-los-salarios}}

\emph{``In general, workers are more likely to support unionization when the union organizer can promise a high wage and a small employment loss.''} (Borjas, 2017)

Los primeros estudios sobre los factores que indicen en la densidad sindical partieron de la mano con Commons et al (1918) y Perlman (1928) quienes desde una perspectiva histórica analizaron la emergencia y dinámica de los sindicatos, en relación con las condiciones económico-sociales que se estaban produciendo a finales del siglo XIX. A partir de ese mismo interés, pero con más precisión, desde la mitad de siglo XX el funcionalismo se preguntó por las causas del crecimiento de la densidad sindical (Dunlop, 1948, Shister, 1953), especialmente poniendo énfasis a los efectos económicos que la Gran Depresión (cf.~Davis (1941)) , la Gran Guerra (cf.~Olson, 1965) y el New Deal (cf.~Bernstein, 1955) podrían tener en la razón que tendrían los trabajadores para sindicalizarse. En consecuencia, en la época emerge un gran paradigma que analiza las causas de la densidad sindical a partir de nociones del ``rational choice''. Estos enfoques se formalizan en teorías de la demanda y oferta por sindicalización, donde la demanda supone que los trabajadores buscan maximizar utilidades y en donde el sindicalismo provee de beneficios y sanciones (incentivos) para la organización. A partir de ahí emergen dos grandes hipótesis: por un lado, las teorías de conciencia en el trabajo plantearán que la densidad sindical crece debido a las malas condiciones de empleo (Perlman, 1928); mientras que, las teorías del control laboral proponen que, en tiempos de auge, la sindicalización aumentará si los sindicatos controlan los despidos y contrataciones (Olson, 1965).
Las teorías que tomaron la delantera son las últimas, y que con el aumento de la inflación durante los 70, comenzaron a mostrar más evidencia de la existencia de un carácter próciclico entre la densidad sindical y los ciclos económicos, es decir, a tiempos de auge más sindicalización.
De manera descriptiva, la Figura 2.7 reporta evidencia del carácter procíclico de la densidad sindical, esto es, en momentos de prosperidad y crecimiento del empleo la masa de sindicalizados aumenta. Es decir, contrario a las teorías que plantean que en momentos de mayor deprivación económica los trabajadores propondrían a organizarse sindicalmente, existen algunas pautas consistentes que indican que el crecimiento de la densidad sindical es próciclico (Calmfors, 2001, p.~19; Riley, 1997). En gran parte de los continentes del globo la tasa de sindicalización y empleo siguen la misma tendencia de variación, a excepción de Latinoamérica que presenta una evidente densidad sindical contracíclica.

\begin{Shaded}
\begin{Highlighting}[]
\CommentTok{#Figura 2.7 Series de tiempo de densidad sindical y tasa de empleo según continente (1960-2020)}
\CommentTok{#Fuente elaboracion propia en base a ICTWSS y OCDE}
\end{Highlighting}
\end{Shaded}

\begin{Shaded}
\begin{Highlighting}[]
\CommentTok{#Figura 2.8 Series de tiempo de densidad sindical y tasa de empleo según continente (1960-2020)}
\CommentTok{#Fuente elaboracion propia en base a ICTWSS y OCDE}
\end{Highlighting}
\end{Shaded}

En estos modelos de demanda y oferta por densidad sindical, las variables del ciclo económico consideradas fueron los \textbf{salarios reales}, \textbf{inflación}, \textbf{crecimiento del empleo} y \textbf{nivel de desempleo}. Las variables del ciclo económico consideradas por la literatura de este periodo son principalmente salarios reales, inflación, crecimiento del empleo y nivel de desempleo. Bain y Elsheik (1976) muestran para Suecia, Estados Unidos, Australia y el Reino Unido evidencia de una relación positiva entre salarios reales y densidad sindical, lo que ha sido interpretado como producto del llamado ``efecto de crédito'', es decir, que un alza en los salarios puede hacer que los trabajadores se sindicalicen si imputan tales ascensos a la acción de los sindicatos y esperan que apoyándolos les vaya aún mejor en el futuro. A su vez, la inflación ha sido incorporada por su ``efecto amenaza'', es decir, pueden hacer que los trabajadores se sindicalicen cuando los precios suben para defender su nivel de vida (Ashenfelter \& Pencavel, 1969; Bain \& Elsheikh, 1976). A su vez, ambos autores muestran una relación positiva con el cambio en el empleo, interpretada como un aumento en la fuerza de trabajo disponible para la actividad sindical y una disminución de la amenaza de sustitución ante la presencia de mercados laborales más expandidos (o en palabras de Wright (2000), un aumento en el poder estructural de mercado). En relación al punto del desempleo, Bain y Elsheikh (1976) muestran que hay una relación negativa entre la desocupación y la densidad sindical, lo que los autores traducen en que mayores niveles de desempleo puede influir en la pérdida de poder de negociación de los trabajadores frente a los empleadores (por las razones indicadas previamente), haciendo menos atractivo el sindicato. En contraste a esta evidencia próciclica, se reportan dos tendencias contrariadictorias: la primera con el desempleo, pues un crecimiento de este puede implicar también un aumento del descontento de los trabajadores llevándolos a unirse más a sindicatos (Ashenfelter \& Pencavel, 1969, p.~437); la segunda, se prueba la hipótesis del ``efecto de saturación'' donde a medida que mayor sea la proporción sindicalizados, más difícil será aumentar la afiliación sindical, pero también se produce el ``efecto aplicación'' donde a mayor densidad sindical previa, los sindicatos se ven con más capacidad de persuasión (Bain \& Elsheikh, 1976).

Como se puede notar, uno de los principales problemas está en que los modelos cíclicos principalmente interpretaron sus resultados en términos de la decisión individual de los trabajadores (Schnabel, 2013a), tratando de formalizar las hipótesis de sus modelos con un correlato con la demanda por sindicalización, donde los individuos decidirían en base a beneficios y costos que estos interpretarían en base a la performance macroeconómica. En consecuencia, los cambios en la afiliación sindical habrían sido analizados más como cambios en la propensión a la sindicalización de los individuos que como cambios en la densidad sindical de los países. En esa línea, Visser (1990) indica que el interpretar el crecimiento de los sindicatos como la agregación de la decisión individual de los trabajadores, el papel de agentes macroeconómicos como los grupos empresariales, los sindicatos y el Estado pueden ser subestimados. También, Schnabel (2002) indica que estos modelos no pueden explicar las diferencias de densidad sindical entre los países, tanto en los distintos niveles como en su transformación, en parte porque no consideran explicaciones estructurales e institucionales de los países para identificar sus diferencias (cf.~Freemand and Pelletier, 1990; Disney, 1990; Stepina y Fiorito (1986); Armingeon, 1989; Schnabel, 1989).

Si bien estos enfoques tuvieron un importante potencial explicando el auge sindical, fueron profundamente ciegos al género. Los estudios de la época indicaron una relación negativa entre el sexo y la densidad sindical, principalmente centrado en que las mujeres eran una fuerza de trabajo temporal o que representaban el segundo ingreso de sus hogares (Kornhauser, 1961; Moore \& Newman, 1975, p.~436). Solo Scoville (1971) Kessler-Harris (1975), y más tarde, Sutton (1980) rechazan estás hipótesis mostrando evidencia que si bien en términos agregados las mujeres tienen menos membresía sindical, si se considera la participación en la fuerza de trabajo tienen la mismas probabilidades que los hombres. Ahora bien, es posible notar que no existe una respuesta en los mismos términos que la densidad sindical general (i.e considerando variables como salarios o desempleo) al porqué la densidad sindical de mujeres es más baja que la de hombres para la época, y solo se propone por parte Sutton (1980) que esto es producto de factores institucionales discriminatorios.
Solo con el auge de la sindicalización femenina a finales de 1980, aparecen investigaciones que muestran evidencia sobre la relación cíclica entre la tasa de empleo y sindicalización femenina (Boston \& O'Grady, 2015). Con las crisis económicas ocurridas durante los 80, una forma de mitigar la pobreza producida en los hogares por la disminución de salarios reales y aumento el desempleo masculino, según Killingsworth and Heckman (1986) señalan que las mujeres aumentaron su participación en el empleo sustituyendo el trabajo doméstico por el trabajo en el mercado. Debido a esto, las mujeres aumentaron su potencial de sindicalización pues no son una fuerza de trabajo transitoria (Humphries \& Rubery, 1988).De hecho, Milner (2017) indica indirectamente que con la crisis financiera del 2008 esta hipótesis se intensificaría: hasta entonces se había creído que la mujer servía como ejército de reserva durante las crisis - tal como ocurrió en la Segunda Guerra Mundial- , y en consecuencia no sería fuerza de trabajo durable y menos aun afiliable al sindicalismo (p.192). Pero más bien lo que pasó durante la crisis del 2008 es que, por un lado, las mujeres a largo plazo aumentaron su participación en el empleo a través de trabajos con menores salarios y más flexibles (Karamessini \& Rubery, 2013) -- como modo de amortiguar la crisis -, y por otro, aumentaron también su participación en el sindicalismo. No existe evidencia sobre porqué esto produciría más afiliación, controlando por el aumento de la participación laboral femenina. Ahora bien, a razón de que se involucran a más mujeres durante las crisis para reducir costos laborales, es posible que esto se exprese en reducciones salariales y deregulación de jornadas, y con ello de igual manera desagravios entre las asalariadas (Cox et al., 2007, p.~719). En consecuencia, la presente hipótesis contiene elementos procíclicos y de las teorías de la conciencia del trabajo para las teorías de densidad sindical generales:

\(H_{1}\): A corto plazo se espera que, en periodos de alto aumento del desempleo, el desempleo masculino va a aumentar y el de mujeres disminuir. Las mujeres aumentarán su sindicalización debido a que el aumento de su empleabilidad estará asociada a puestos precarios y con brechas salariales altas, lo que las llevará a sindicalizarse como un modo de resguardar sus condiciones laborales. En consecuencia, aumento en el desempleo femenino feminiza la densidad sindical debido a la inclusión de más mujeres en los sindicatos, pero también al escape de los hombres de estas por el aumento de su desempleo.

\hypertarget{determinantes-estructurales-del-declive-sindical-el-efecto-de-la-globalizaciuxf3n-y-desindustrializaciuxf3n}{%
\section{Determinantes estructurales del declive sindical: el efecto de la globalización y desindustrialización}\label{determinantes-estructurales-del-declive-sindical-el-efecto-de-la-globalizaciuxf3n-y-desindustrializaciuxf3n}}

A partir de 1980 la densidad sindical sufre un quiebre con su tendencia al alza de los años anteriores, sobre todo, debido a los procesos de desindustrialización y cambio tecnológico, que producen una reorganización de la producción (Jenkins y Leich, 1997) y circulación del capital (Streeck y Schmitter, 1986), lo que tuvo como consecuencia la erosión de las bases tradicionales del movimiento obrero (Silver, 2003). Desde estos años la estructura económica y la composición de la fuerza de trabajo va a transitar desde una estructura productiva industrial principalmente dominada por hombres, hacia una de servicios en donde las mujeres comienzan a tener un mayor protagonismo en la fuerza de trabajo.
En la figura 2.X. se puede observar el cambio, a nivel mundial, en las estructuras económicas de los países, resaltando el crecimiento del sector servicios en todas las regiones, llegando a más del 50\% de los ocupados; la desindustrialización progresiva de Europa y Asia del Este desde los años noventa; la industrialización de Asia del Este y Pacífico, y África, después de los años dos mil; y la baja generalizada del sector de la agricultura. En solo 20 años el sector servicios creció cerca de un 15\% tanto en países OCDE como en economías en vías de desarrollo (Banco Mundial, 2019)

\begin{Shaded}
\begin{Highlighting}[]
\CommentTok{#Figura 2.X Empleo según sector económico por continente (1990-2019)}
\end{Highlighting}
\end{Shaded}

\begin{Shaded}
\begin{Highlighting}[]
\CommentTok{#Figura 2.5 Participación en el trabajo asalariado y empleo parcial según sexo a nivel mundial (1960-2020)}
\CommentTok{#Otra figura }
\end{Highlighting}
\end{Shaded}

Como se puede evidenciar en la Figura 2.5, entre 1960 a 1980, las cifras indican que el ingreso importante de las mujeres a la fuerza, que, si bien en los años venideros fluctúa considerablemente, ya en 1980 se ve una constante creciente. Ahora bien, esta tendencia estructural de cambio evidencia diferencias importantes en el globo: entre 1980 a 2010 se dio un aumento en promedio de 54\% a 71\% de participación femenina en el empleo, mientras que en América Latina la variación se dio desde 41,2\% a 49,9\%. (cf.~Goldin \& Katz, 2016; ILO, 2010, p.~29). Ya para el 2000 un 85,1\% de las mujeres ocupadas estaban concentradas en el sector servicios (ILO, 2004). Las ocupaciones desarrolladas en el sector financiero, comunicaciones, salud, servicios sociales y comunitarios y administración pública se fueron feminizado mientras que las mujeres se fueron subrepresentando en el sector industrial (Gálvez, 2001, p.~66). Con esto se quiere indicar que las mujeres tienen un patrón diferenciado de ingreso al mercado laboral, donde si bien la participación laboral femenina ha aumentado, en general el acceso ha sido limitado tipo de trabajos que se concentran en el sector servicios, en empleos de bajos salarios y con jornadas flexibles sin seguridad social malas condiciones de trabajo y ausencia de protección social. De hecho tal como nos muestra la Figura 2.X el trabajo part time sigue siendo más prevalente en mujeres que en hombres a nivel mundial (Fagan et al., 2015). El informe de Panorama Laboral (ILO, 2018) de la OIT muestra que 68 de los 73 países estudiados presentan una feminización del trabajo part time, junto con que el 14\% de las mujeres empleadas son a tiempo parcial, mientras que los hombres solo un 7\%. Así también, menos de la mitad de las mujeres tienen algún empleo a tiempo completo, mientras que más de un 75\% de los hombres empleados trabajan a tiempo completo.

Con estos cambios antes indicados, la literatura de densidad sindical cambió su enfoque. Básicamente, los estudios se centraron apuntar cómo estos cambios en la composición y condiciones del empleo erosionarían al sindicalismo, particularmente fundamentado en cómo, por ejemplo, las mujeres y los trabajadores a tiempo parcial tendrían menos propensión a sindicalizarse (Ebbinghaus, et al 2008). Bajo este esquema, una serie de estudios se centraron en analizar las causas del declive sindical en los distintos países (Walerstein and Western, 2000), bajo un enfoque de determinantes estructurales de la sindicalización (Schnabel, 2002; Ebbinghause y Visser, 1999). La literatura estuvo centrada principalmente en factores medidos transversalmente y que se reducen a características sociales de quienes participaban en sindicatos y quienes no (eg. edad, sexo, raza y educación), y por ello, más que modelos de densidad sindical referencian a modelos individuales de propensión u oportunidad de membresía sindical (cf.~Scoville, 1971; Bain and Elias, 1985; Berg and Groot, 1992; Fitzenberger et al., 1999). Sin embargo, un número no menor de estudios estudiaron de manera agregada los cambios en la estructura productiva (véase Figur 2.X) y en la composición del empleo (véase Figura 2.X) principalmente a través de variables asociadas a la tercerización y feminización, indicando que estas producirían una merma en el poder de los trabajadores.

En primer lugar, según Polachek (2004) el declive sindical está influenciado por tendencias de cambio en la estructura industrial. Una serie de investigaciones apoyan esta proposición indicando que, si la proporción del sector económico de servicios aumenta en desmedro del manufacturero, la densidad sindical va a disminuir (Blaschke, 2000; Lee, 2005; Polachek, 2004; Carruth and Schnabel, 1990 for Germany; Farber and Western, 2001 para USA). Si bien se ha indicado que esta variable no es significativa (Brady (2007), Blanchflower y Bryson (2009), Charlwood y Haynes (2008; y Fitzenberger et al (2011)), Vachon et al (2016) con técnicas más avanzadas se comprueba que el crecimiento del sector servicios tendría un efecto negativo sobre la densidad sindical bajo el argumento de que en el sector terciario los sindicatos tienen más dificultades para reclutar socios por las condiciones flexibles de trabajo a las que se enfrenta (Traxler, 1999; Wrigley \& Lowe, 2010). Del mismo modo, el aumento de la proporción de empleados en el sector público ha sido interpretado como un factor que aumenta la densidad sindical, bajo el entendido de que los costos de sindicalización para empleados (como despidos y represalias) son menores en este sector (Schnabel, 2003), y por lo mismo la densidad sindical es más alta en este sector (Ebbinghaus, 200). De hecho, Kirmanoğlu y Başlevent (2012) muestran evidencia para 24 países donde si el empleo público aumenta en el tiempo, la densidad sindical también aumentará\footnote{Esta variable debe ser mirada con detención. En primer lugar, tal como señala Schnabel (2013) dista de ser una variable con tendencia estable en la medida en que en muchos países primero aumentó el empleo público con la expansión de los Estados de Bienestar, pero luego disminuyó con la privatización y la desregulación. A su vez, en países como Chile, los empleados públicos no tienen derecho a sindicalización por lo que son excluidos de las tasas de densidad sindical.}.
En segundo lugar, si crecen los grupos que se espera que tengan menor adhesión a la fuerza de trabajo-- y por tanto sean más difíciles de organizar (Ebbinghaus et al., 2011; Schnabel and Wagner, 2007; Visser, 2006)- , es probable que la densidad sindical descienda. Otro argumento desarrollado indica que si los grupos menos ``militantes'' con el sindicalismo crecen, esto es grupos que no pertenecen a la base obrera tradicional como las mujeres, el sindicalismo irá en declive(Sinclair, 1995; Tomlinson, 2005). Los estudios más recientes apuntan a como el aumento de las mujeres, proporción de extranjeros y empleados atipícos como los trabajadores a tiempo parcial e informales disminuyen la densidad sindical (Visser, 2012; Ebbinghaus et al., 2011; Schnabel and Wagner, 2007; Visser, 2006).

Empero, las explicaciones estructurales hasta aquí expuestas nuevamente ponen el foco en razones individuales (eg. grupos con menor disposición militante a la sindicalización) para explicar el declive de la sindicalización a nivel agregado. El problema no es puramente formal: gran parte de los estudios de densidad sindical indican una relación negativa con el ser mujer, y con ello el aumento de la membresía sindical -controlando por su mayor participación en el mercado- no podría ser entendida. Esta misma limitante ocurre en otros contextos donde no necesariamente los cambios en la estructura económica y la composición de la fuerza de trabajo produjeron un impedimento a la organización sindical (Schnabel, 2013b), incluso apareciendo nuevas tendencias donde no declina la sindicalización sino que emerge la ``tercerización del conflicto laboral'' (Kocsis et al., 2013).

En cierto sentido los estudios de Haile (2016) para Inglaterra son de los pocos de este tipo que nos permitirían entender porque la sindicalización se feminiza. Desde 1980, la fuerza de trabajo femenina ha aumentado notablemente de la mano con la expansión del sector servicios, cambiando significativamente la composición de género de sectores. La literatura plantea que esto puede producir que la estructura sindical, con la incorporación de mujeres, se va a ver enfrentada a un antagonismo con la estructura tradicional sindical dominada por los hombres, y por ello podría ocurrir también un ``escape'' de los miembros masculinos (Haile, 2017). En la suma y resta esto podría producir la feminización de la densidad sindical pues si más mujeres entraran a la estructura sindical robustecida por mujeres, más hombres se irán de la organización al verse menos representado en estas organizaciones.

Una forma de aproximarnos a nivel de los países a este fenómeno es viendo el efecto que tiene el crecimiento de sectores que están feminizados, como el sector servicios; o el crecimiento de empleos feminizados, como el empleo partime. En consecuencia, se explora si el crecimiento de variables asociadas al empleo pink collar (i.e mujeres trabajadoras de los sectores servicios, de empleos flexibles y con bajos salarios) producen una mayor feminización de la densidad sindical. Esta hipótesis exploratoria es plausible sobre todo a la luz de la convergencia entre: (1) la tendencia hacia la feminización del empleo remunerado ha ido al alza, (2) el sector servicios y empleo par time\^{}3 \footnote{Con el avance de las mujeres, los sindicatos también se han abierto a sindicalizar más empleados a tiempo parcial. Las estimaciones de la OECD (2017) muestra que un 16,7\% de los empleados en las economías desarrolladas trabajan part-time, siendo un 70\% de ellas mujeres. La participación de los empleados parciales en el sindicalismo aumentó con una variación en un 2\% anual (2016 a 2018), pero que si bien es lento la brecha se vuelve cada vez más pequeña.} ha ido en expansión -siendo este la principal inserción de las mujeres, (3) la ``tercerización de los conflictos laborales'' (Kocsis et al., 2013) y (4) feminización densidad sindical, lo que para Artus y Plüger (2015, p.~93) es como sumar uno más uno plantear alguna hipótesis sobre la relación entre la feminización de los conflictos laborales y estas tres tendencias.

\(H_{2}\): Con el crecimiento de los sectores económicos feminizados, como el sector servicios, a largo plazo se producirá un cambio en la estructura sindical que por un lado se va a producir un aumento de la densidad sindical femenina y por otro un escape de la membresía masculina. En suma y resta, la sindicalización se feminizará.

\hypertarget{determinantes-institucionales-de-la-densidad-sindical}{%
\section{Determinantes institucionales de la densidad sindical}\label{determinantes-institucionales-de-la-densidad-sindical}}

Si bien los últimos 20 años la densidad sindical ha disminuido en 38 de los 53 países a analizar (Tabla 2.1 ); en 12 naciones la sindicalización ha aumentado o se ha mantenido igual en los últimos años. De hecho, en tendencias a largo plazo solo 5 en cada 9 países han disminuido su densidad sindical (1960 a 2019). Con esto, la evidencia es clara en que desde los años 2000 existe una divergencia de trayectorias sindicales entre los países. Mientras en algunos países la densidad sindical mantiene su declive, en otros se estanca y, en otros, sorpresivamente aumenta (Figura 2.2). A su vez, no solo se han diferenciado diferencias temporales (Figura 2.1). Investigaciones recientes han constatado patrones diferenciados en la densidad sindical entre países anglosajones y del Europa Occidental (Schmitt and Mitukiewicz, 2012; Visser, 2006; Scruggs, 2002; Chechi y Lucifora, 2002): por un lado, podemos encontrar países con densidades sindicales bajo un 12\% (Francia, Corea del Sur, Estados Unidos) mientras que encontramos países con afiliación sindical por sobre un 60\% (Finlandia, Suecia, Dinamarca, etc)

\begin{Shaded}
\begin{Highlighting}[]
\CommentTok{#Tabla}
\end{Highlighting}
\end{Shaded}

Una de las dificultades que implícitamente se ha abordado hasta ahora es que los análisis de series temporales y comparados han tenido poca capacidad de integración. Schnabel (2013) puntualiza que este problema ha sido abordado considerando factores institucionales en el análisis, factores tales como contexto de las relaciones industriales, la legislación laboral y las orientaciones sociopolíticas de los gobiernos. De este modo, los estudios que abordan los determinantes de los cambios en la densidad sindical han apuntado a los marcos institucionales de los países para poder entender sus dinámicas y diferencias.
En primer lugar, Cecchi y Visser (2005) y Visser (2006) indican que, si bien el desempleo tiene un efecto negativo sobre la densidad sindical en el corto y largo plazo, esa relación puede ser puesta en duda en los países en donde se provee de seguridad al desempleo (Brady (2007), Blanchflower y Bryson (2009), Charlwood y Haynes (2008); y Fitzenberger et al (2011)). De hecho, Ebbinghaus, Göbel y Koss (2011) en un estudio multinivel apuntan a que los sistemas Ghent explican por qué la densidad sindical crece a pesar de que los niveles de desempleo pueden ser mayores, comprobando lo que previamente ya Olson (1965) había hipotetizado con su teoría del control del trabajo. El argumento principal está en que en países donde el desempleo está protegido por políticas sindicales, las personas buscarán sindicalizarse de modo de resguardarse ante posibles periodos de receso económico (Ebbinghaus and Visser, 1999; Olson, 1965; Rothstein, 1992; Western, 1997), mientras que en países donde ese tipo de protecciones no existe o no son controladas sindicalmente, el desempleo sigue su impacto ya reseñado. De manera similar, Scrugg y Lange (2002) encuentran evidencia de que las estructuras centralizadas de negociación colectiva producen una mayor sindicalización en escenarios donde los salarios reales disminuyen, como una forma de resolver el conflicto salarial entre capital y trabajo.\\
Un tercer punto analizado y más controversial tiene que ver con la globalización: la apertura del comercio, mayor movilidad de flujos de capital y el aumento de inversión extranjera han sido interpretadas como factores que socavan la densidad sindical (Lange et al.~1995; Ebbinghaus y Visser, 2000). Wallerstein y Western (2000) indicaron que un aumento en la globalización, medido en aumento de importaciones, disminuye la densidad sindical debido a que la globalización empuja a las economías nacionales a ``una carrera hacia el abismo'' (Silver, 2003) produciendo efectos negativos sobre las condiciones de los trabajadores (Stiglitz, 2002; Rodrik, 2011) y haciendo más difícil su organización. La competencia entre las nacionales llevó a implementar formas de trabajo flexible para reducir los costos laborales, entre ellas, el empleo parcial y la externalización. Western (1997) y Blaschke (2000) reportan que políticas de apertura del comercio y aumento de la inversión extranjera disminuyen la probabilidad de sindicalización, sobre todo a partir de 1980. Sin embargo, un análisis multinivel de 18 países, Brady (2007) encuentra evidencia de que estas variables no tienen un efecto significativo si se controla por variables del ciclo económico, tal como lo realiza Scrugg y Lange (2002).
Estos resultados muestran que factores asociados a la globalización y financiarización (eg., aumento de importaciones, crecimiento del sector servicios, movilidad del capital) fueron puestos en duda debido a que los resultados distan de ser iguales en todos los países (Brady, 2007). Por un lado, las teorías de la convergencia plantean que a partir de 1980 todos los países han tendido hacia una trayectoria neoliberal evidenciada en desregulación y reconversión institucional, lo que produce como resultado una debilidad generalizada en los sindicatos y un declive en el conflicto industrial, expresada en un descenso de la densidad sindical (Baccaro \& Howell, 2011, p.~529)
Por otro lado, teorías del corporativismo (Crouch, 1993), o más recientemente, teorías de las variedades del capitalismo (VoC) plantean que existirán distintos resultados a la globalización debido a que estos dependen del marco institucional en el cual se desarrollan estos cambios (Hall \& Soskice, 2001). El enfoque VoC indica que en economías coordinadas de mercado (CMEs) el marco institucional va a favorecer la afiliación sindical, en la medida en que se busca asegurar el reconocimiento de los trabajadores. Más específicamente, Thelen (2001) muestra evidencia de que la densidad sindical en países CMEs es más alta a pesar de los altos niveles de desempleo, en contraste con los países anglosajones asociados a LMEs.
En la misma línea, las teorías neo-corporativistas indicarán que en sistemas en donde se promueve la coordinación salarial tripartita (sindicatos, empresarios y gobierno) la densidad sindical también aumentará (Western, 1997;Ebbinghaus y Visser, 1999; Brady, 2007; Hechter, 2004). Las teorías de los recursos de poder también apoyarán estos supuestos (Korpi,1983): países donde se producen condiciones más amigables para los sindicatos, como por ejemplo ante la presencia de gobiernos de izquierda, se promoverá relaciones laborales ``pro-trabajador'' y por ello, la afiliación sindical no se verá mermada con políticas que restringen o dificultan la participación sindical Brady, 2007; Korpi, 1983; Wallerstein, 1989; Western, 1997)
Para América Latina el puzle entre factores cíclicos, estructurales e institucionales debe ser mirado con detención. Principalmente existe un gran vacío de estudios sobre los determinantes de la densidad sindical en el continente, más aún considerando sus factores a nivel macro. Ahora bien, siguiendo el esquema de VoC, Schneider (2013) propone una tipología para América Latina que consiste en un sistema de relaciones laborales jerárquicas y segmentadas cuyos resultados destacan la baja cobertura de la negociación colectiva, sindicatos débiles en cuantitativa y cualitativamente y una gran fuerza de trabajo informal

\begin{Shaded}
\begin{Highlighting}[]
\CommentTok{#Figura VOC densidad sindicaly salario minimo}
\end{Highlighting}
\end{Shaded}

La Figura 2.X muestra la posición de cada uno de los países según su densidad sindical y el salario mínimo real (en USD\$) que lograron coordinar para el último año. Como se puede identificar existe una clara agrupación según tipo de países a partir de los salarios que logran coordinar -- siendo los más altos en CMEs y más bajos en LMEs -- y la densidad sindical que estos países tienen en la actualidad.
Tal como ocurre con la sindicalización en general, la feminización de la densidad sindical ha ocurrido en algunos países y en otros no. Si bien una parte de ello se puede deber a las diferencias que presentan entre los países en términos de la participación del sector servicios o las tasas de desempleo, es de esperar que la feminización de la densidad sindical esté asociada al marco institucional en el que se desenvuelven las relaciones laborales. Howell (1996) indica que la negociación de salarios mínimos y protección del trabajo es uno de los elementos más importantes para entender la relación entre mujeres y sindicatos. A simple vista no es una cosa tan distinta para los hombres, pero Rubery y Fagan (1995, p.~521) plantean que aspectos asociados a la negociación como la cobertura a negociación colectiva es un aspecto institucional relevante para la densidad sindical femenina: hasta los 80 los sectores económicos que en términos de empleo habían sido tradicionalmente dominados por las mujeres (i.e sectores informales), no tenían derecho a la negociación colectiva y negociaban principalmente a través de Consejos de los Salarios. A través de estos mecanismos de diálogo salarial las trabajadoras no tenían necesidad de sindicalizarse, mientras que cuando empiezan a tener participación más directa en empleo formal la extensión o no de derechos colectivos para a ser un factor relevante para determinar los niveles de sindicalización. Conectando con la discusión anterior sobre la tipología de países, se ha dicho que en los países de economías coordinadas poseen mayor extensión de beneficios colectivos mientras, en contraste con países liberales. En consecuencia,

\(H_{3}\): Las instituciones laborales como la coordinación salarial y la extensión de la negociación colectiva mediarán el efecto del aumento de la participación laboral y desempleo femenino, en la medida en que, si la participación laboral femenina aumenta y no hay presencia de coordinación salarial ni extensión de beneficios, más mujeres se sindicalizarán de modo de mejorar sus condiciones de trabajo. Es decir, es esperable que en economías liberales se esté produciendo una feminización de la sindicalización.

\hypertarget{metodologuxeda}{%
\chapter{Metodología}\label{metodologuxeda}}

\hypertarget{data-y-variables}{%
\section{Data y variables}\label{data-y-variables}}

Los datos usados en esta investigación provienen de la construcción de una base de datos longitudinal (FDL\footnote{Fábrica de Datos Laborales o FDL es una base de datos de construcción reproducible. Por ello la construcción, validación y procesamiento de FDL responde a los criterios IPO- TIER Project y está libre en el repositorio de esta tesis. A su vez, la base de datos se actualiza cada vez que las fuentes principales de información actualizan la información que suben a sus sitios web. A su vez se puede encontrar en el siguiente link el libro de códigos donde se especifica la medición de cada una de las variables.}) sobre relaciones laborales a partir de la unión de fuentes como ICTWSS, OCDE, ILO y Banco Mundial, que proveen información temporal a nivel de los países. A esto se agrega la estimación y validación de los dos indicadores principales para este estudio, que son la densidad sindical femenina y masculina, que fueron construidas a partir de encuestas de hogares, encuestas a fuerza de trabajo y/o datos administrativos para los países en estudio (ver Apéndice C). FDL es una base de datos longitudinal que contiene 3360 observaciones y 340 variables laborales, económicas e instituciones a partir de información desde 1960 a 2018 de 56 países del mundo. Como se puede evidenciar en la Tabla 3.1, gran parte de estos países corresponden a países OCDE, y aquellos que no son países OCDE pertenecen al G20.

\begin{Shaded}
\begin{Highlighting}[]
\CommentTok{#Tabla Tabla 3.1 LatinoaméricaArgentina, Brazil, Chile, Colombia, Costa Rica, Mexico, Uruguay; de Norte América, Estados Unidos y Canadá; de Europa, Austria, Belgium,Bulgaria,Croatia, Cyprus, Czech Republic, Denmark, Estonia, Finland, France, Germany,  Greece,Hungary, Ireland, Italy, Latvia, Lithuania, Luxembourg,Malta, Netherlands, Norway, Poland,Portugal,Romania, Slovak Republic, Slovenia, Spain, Sweden, Switzerland, United Kingdom; de Oceanía, Australia, Iceland,New Zealand; de Asia, China, Hong Kong, China, India, Israel, Japan, República de Korea, Malaysia, Russian Federation,    Singapore, Taiwan, China,Turkey, Indonesia, Philippines; y de África, Sudáfrica. }
\end{Highlighting}
\end{Shaded}

El periodo de análisis abarca entre los años 1980 y 2018, por las mismas razones teóricas y empíricas indicadas por Vachon et. al (2016). Primero, no existe disponibilidad de datos para todas las variables rezagadas antes de 1980 -- solo 10 países tienen datos desde 1960. Segundo, desde 1980 se reconoce el inicio del periodo neoliberal del desarrollo capitalista (Baccaro \& Howell, 2011; Tomaskovic-Devey et al., 2015) lo que permite evidenciar los inicios del proceso de crecimiento del sector servicio, desregulación del mercado laboral y divergencia del papel del Estado en las relaciones laborales. Tercero, es importante notar que el intervalo de tiempo permite considerar el efecto de la crisis económica de 2008 en la densidad sindical.

\hypertarget{variable-dependiente}{%
\subsection{Variable dependiente}\label{variable-dependiente}}

\emph{\(FUDI_{i}\)}
La variable dependiente en esta tesis es la feminización de la densidad sindical (Feminization Union Density Index-FUDi), que es un índice que mide la división sexual de la densidad sindical mediante la relación de las densidades sindicales por sexo en una única dimensión, es decir, la afiliación sindical neta de mujeres y hombres como proporción de los asalariados en el empleo. Esta variable fue estimada y validada en base a dos indicadores principales para este estudio, que son la densidad sindical femenina y masculina, que fueron construidas a partir de encuestas de hogares, encuestas a fuerza de trabajo y/o datos administrativos para los 56 países en estudio (para más detalles de fuentes y consistencia ver \emph{Apéndice}).

\[FUD_{i} = \frac{\frac{NUM_{f}*100}{WSEE_{f}}}{{\frac{NUM_{m}*100}{WSEE_{m}}}} = \frac {UD_{f}}{UD_{m}}\]
Donde,

\begin{itemize}
\tightlist
\item
  \(NUM\): Número de afiliados en organizacioones sindicales
\item
  \(WSEE\): Número de asalariados del total de la fuerza de trabajo
\item
  \(UD\): Densidad sindical
\item
  El subíndice \(f\) indica femenino y \(m\) masculino
\end{itemize}

Los valores de FUDi oscilan entre 0,26 a 2,12, mediana 0,9516 y desviación estándar 0.3416. Cuando FUDi es cercano a 1, indica que la densidad sindical femenina y masculina son iguales. Mientras que cuando FUDi es sobre 1 expresa que la densidad sindical femenina es mayor a la masculina. En ese caso se habla de una densidad sindical femenizada.

\hypertarget{variables-independientes}{%
\subsection{Variables independientes}\label{variables-independientes}}

\emph{Determinantes Cíclicas}

Para capturar los efectos del ciclo económico sobre la densidad sindical se probará la relación con el desempleo y los salarios reales, que son obtenidos por la base LFS (Labor Force Statistics) recopiladas por la OIT y OCDE respectivamente.
El desempleo se mide como el porcentaje anual de desocupados del total de la fuerza de trabajo, dejando afuera a ocupados e inactivos del cálculo. Los salarios mínimos reales se miden por hora y anualmente, deflactadas por los índices nacionales de precios al consumidor. Los datos de salarios mínimos se convierten en una unidad monetaria común que utiliza las Paridades de Poder Adquisitivo (PPA) en dólares, con el objeto de medir el poder de compra efectivo de los salarios.
Determinantes Estructurales
El empleo en el sector servicios se define como la tasa anual de la fuerza de trabajo en el sector servicios, es decir, en las actividades de las secciones económicas que van de la G a la U en la Clasificación Industrial Estándar Internacional (cf.~ISIC rev.4 2008\,; Apéndice C) El tamaño de las actividades productivas proviene de la estimación de OIT y son recopiladas por el Banco Mundial.
La participación laboral femenina es calculada como la fuerza de trabajo femenina dividida por el total de mujeres en edad de trabajar, la cual depende de los países, pero en general esta va desde los 15 a los 64 años.
La incidencia del empleo parcial corresponde al porcentaje de trabajadores parciales en el total de los asalariados de cada país. Esta es una medida que varía según la definición del empleo parcial para cada país, pero en general se define por jornadas que van entre 30 a 35 horas por semanas. Tanto la participación laboral femenina como la incidencia del empleo parcial provienen de LFS que es recopilada por OCDE.

\hypertarget{variables-de-interacciuxf3n}{%
\subsection{Variables de interacción}\label{variables-de-interacciuxf3n}}

\emph{Determinantes Institucionales}

Para poder distinguir entre economías coordinadas y liberales, se utilizará la variable coordinación salarial construida por Visser (2019) en ICTWSS. En términos teóricos, la variable se define como grado de coordinación institucional que hay sobre el mercado laboral (Kenworthy, 2001, 2003), donde 1 indica una negociación localizada en las firmas y 5 indica una negociación centralizada y vinculante. En consecuencia, la coordinación salarial es una forma de medir la centralización y coordinación de las políticas laborales, y evidencia reciente demuestra un patrón claro de alta coordinación salarial en CMEs y baja en LMEs (Baccaro \& Howell, 2011; Hassel, 2015; Kim et al., 2015). En términos prácticos, coordinación salarial fue escogida pues dentro de las medidas de centralización, es de las variables que tiene mayor disponibilidad de datos para un mayor número de países.

\hypertarget{variables-de-control}{%
\subsection{Variables de control}\label{variables-de-control}}

Se controlará por variables que han sido tradicionalmente utilizadas en los estudios de sindicalización. Primero, las variables de control sindicales son la densidad sindical previa como una forma de controlar el efecto anterior de la variable dependiente sobre sí misma y la cobertura de la negociación colectiva, porque como se ha indicado la densidad sindical de algunos países muestran una alta correlación con esta variable -- a excepción de Francia. Segundo, si bien la literatura tradicional considera variables del ciclo económico control, como en esta investigación son variables sustantivas, solo se controlará por crecimiento económico.

\hypertarget{modelo}{%
\section{Modelo}\label{modelo}}

Esta investigación utiliza un análisis de series de tiempo y transversal para 56 países durante 1980 a 2018, o 2128 países-años. Se analizan los efectos que tienen los determinantes cíclicos, estructurales e institucionales en la feminización de la densidad sindical con un modelo de corrección de error (ECM). La razón principal tiene que ver con que estos modelos capturan los efectos a corto y a largo plazo de las variables, pudiendo distinguir entre ellas las variables ``cíclicas'' y las ``estructurales'', respectivamente (Checchi \& Visser, 2005; Disney, 1990). A su vez, ECMs permiten modelar efectos con variables no estacionarias y cointegradas (Beck \& Katz, 2011) , esto es establecer relaciones lineales con variables que tienen tendencias (de ascenso o descenso) a largo plazo pero integrando también las de ciclo o coyuntura (Checchi \& Visser, 2005)

La literatura especializada sobre densidad sindical ha planteado que el ECM permite responder a las explicaciones de cambio(cf.~Carruth \& Disney, 1988; Disney, 1990), pero también de diferencias entre países (Schnabel, 2002), principalmente porque permite identificar una serie de especificaciones \footnote{Esto se refiere a que no existe una clarificación de la cadena causal-temporal de las variables. Los modelos han determinado que el salario determina a la densidad sindical, pero no se considera como la inflación pueden mediar esa relación dentro del ciclo económico, y, en consecuencia, se pasa por alto las teorías de la inflación del empuje salarial. También, no es claro si los modelos empíricos es el nivel o tasa de desempleo la que juega el rol de inhibir el crecimiento sindical. Otro problema se deriva de la especificación de los salarios debido a que, si los precios y salarios aumentan al mismo ritmo, los salarios reales no cambiarían, entonces el ``efecto crediticio'' y ``efecto amenaza'' deberían modelados a partir de los salarios reales o controlando por la inflación. En consecuencia, si los beneficios sindicales son un bien normal, solo un aumento de los salarios reales debería aumentar la demanda por sindicalización.} de variables para distintas explicaciones teóricas (Keele et al., 2016)

En términos procedimentales para corregir la heteroscedasticidad producida por que la varianza de los errores difiere por cada uno de los países, se emplean errores estándares especificados como panel (cf.~Beck \& Katz, 1995) y una corrección autorregresiva de primer orden (Vachon et al., 2016), esto es un modelo aleatorio en donde se estima la media de la variable dependiente (FUDi) según la media de sus observaciones pasadas.

A su vez, se han incluido en los modelos una serie de efectos fijos para mostrar la invariabilidad de algunas variables por países (e.g institucionales, que por lo general son estables), lo que permite que las diferencias entre países de deban más a la baja varianza dentro del país, que a diferencias no observadas entre países -- por casos perdidos. Para aquellas variables que tienen una mayor variación se ha realizado imputaciones múltiples, es decir, se han rellenado los casos perdidos a partir de procesos estocásticos -- considerando estructuras de pérdidas de dato de tipo MCAR (Young \& Johnson, 2015).

En relación a las ecuaciones estimadas, se parte desde un modelo de punto muerto (cf.~dead start model en Keele et al., 2016), que permite establecer un punto de referencia en el tiempo,. Esto desde el supuesto de que a largo plazo existe un equilibrio pero que es posible distinguir tendencias a corto plazo y a largo plazo. La ecuación de punto muerto se específica en la \emph{Formula 3.1} \footnote{Esto se refiere a que no existe una clarificación de la cadena causal-temporal de las variables. Los modelos han determinado que el salario determina a la densidad sindical, pero no se considera como la inflación pueden mediar esa relación dentro del ciclo económico, y, en consecuencia, se pasa por alto las teorías de la inflación del empuje salarial. También, no es claro si los modelos empíricos es el nivel o tasa de desempleo la que juega el rol de inhibir el crecimiento sindical. Otro problema se deriva de la especificación de los salarios debido a que, si los precios y salarios aumentan al mismo ritmo, los salarios reales no cambiarían, entonces el ``efecto crediticio'' y ``efecto amenaza'' deberían modelados a partir de los salarios reales o controlando por la inflación. En consecuencia, si los beneficios sindicales son un bien normal, solo un aumento de los salarios reales debería aumentar la demanda por sindicalización.}, donde

\[\triangle FUDi_{i,t} = \alpha_1,i + t- \beta_1FUDi_{i,t-1}+ \beta_2X_{i,t-1}+ \epsilon_{i,t}\]
Donde,
- \(\triangle FUDi_{t}\) representa la primera diferencia entre \(FUDi_{t}\) y \(FUDi_{t-1}\)
- \(i\) representa cada uno de los países
- \(\alpha_1,i\) representa la desviación en el tiempo para cada país
- \(\beta_1\) representa la tasa de ajuse o correción de errores de \(FUDi\)
- \(\beta_2\): efecto directo de la variable \(X_{i,t-1}\) sobre \(\triangle FUDi\)
- \(X_{i,t-1}\): determinate de la densidad sindical

El modelo se lee como: un aumento unitario de \(FUDi_{t-1}\) conduce a una disminución unitaria de \(\beta_1\) en \(\triangle FUDi_{t}\) y, por lo tanto, seproduce un aumento unitario de \(\beta_1\) en \(FUDi_{t}\), controlando por el resto de las variables. En consecuencia, el efecto acumulado a largo plazo de un aumento unitarario de \(X_{i}\) en \(FUDi\) es la suma de una serie geométrica infinita, donde \(n\) representa el numero de unidades tiempo-países que tienen efectos directos sobre la feminización de la densidad sindical.

\[\sum_{n=0}^{\infty} \beta_2(1-\beta_1)^{n} = 1\]

Reemplazando por la sumatoria, la ecuación final que contiene la predicción de la variación de \(FUDi\) (\(\triangle FUDi\)) a largo plazo se basa en el modelo de Bewley (1979) para densidad sindical:

\[FUDi_{i,t} = \beta_1^{-1}\alpha_1,i + t- \beta_1^{-1}(1-\beta_1)\triangle FUDi_{i,t}+ \beta_1^{-1}\beta_2X_{i,t-1}+ \epsilon_{i,t}\]

Finalmente, con el objetivo de probar la hipótesis de interacción entre participación laboral femenina y crecimiento del sector servicios de Haile (2017) pero ahora nivel nacional, se estimará un modelo final que incluye dicha interacción entre aquellas variables.

\hypertarget{resultados}{%
\chapter{Resultados}\label{resultados}}

\hypertarget{analisis-descriptivo}{%
\section{Analisis descriptivo}\label{analisis-descriptivo}}

\hypertarget{modelo-emc}{%
\section{Modelo EMC}\label{modelo-emc}}

\hypertarget{interacciones}{%
\section{Interacciones}\label{interacciones}}

\hypertarget{ajuste}{%
\section{Ajuste}\label{ajuste}}

\hypertarget{conclusiuxf3n}{%
\chapter{Conclusión}\label{conclusiuxf3n}}

  \bibliography{book.bib,packages.bib}

\end{document}
